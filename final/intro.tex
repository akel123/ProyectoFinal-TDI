\section{Introduction}

Relative entropy has been extensively studied in the past in many different contexts and under a variety of different names, much of which had to do with the change over time between two distributions. However, the main focus of this paper concerns its properties under mappings by stochastic matrices. The results can be used to provide more information on bounding the rates of convergence to equilibrium of ergodic Markov chains and Markov processes. After reviewing some definitions, we will see theorem 3.1 (omiting the majority of the proof) which will be used to prove the the two main results selected for this elaboration: theorems 4.1 and 5.4. Finally, we will see the some of the ideas of this paper in the context of the what we saw throughout the semester.
